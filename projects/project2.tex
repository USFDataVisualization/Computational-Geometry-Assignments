\documentclass[a4paper,12pt]{article}

\usepackage{../usfdvl}


\title{Project 2: Polygon Diagonals}
\SetDocumentFooter{}{}


\begin{document}

\maketitle

\section{Objectives}

\myparagraph{In this assignment you will implement some of the basic polygon functionality.}


\vspace{5pt}
\section{Ground Rules}

\myparagraph{This assignment is intended to be done alone. You may ask others for help with figuring out strategies. However, the code must be yours.}

\vspace{5pt}
\section{Assignment Instructions}

\begin{itemize}

\item Download the provided skeleton code and complete the unfinished functions in \texttt{Polygon.pde}.

\begin{itemize}

   \item \texttt{boolean isSimple( )} --- This function checks to see that the boundary of the polygon is simple.
   
   \item \texttt{boolean pointInPolygon( Point p )} --- Returns true if the point p is inside of the polygon.
   
   \item \texttt{ArrayList<Edge> getDiagonals( )} --- Returns all of the valid diagonals for the polygon.
   
   \item \texttt{boolean ccw( )} --- Returns true if the polygon is oriented counterclockwise.
   
   \item \texttt{boolean cw( )} --- Returns true if the polygon is oriented clockwise.
         
   
\end{itemize}

\item To test your code the Processing skeleton provided gives visual feedback for creating polygons and testing your capabilities.

\end{itemize}


\section{Submission}

\myparagraph{Compress your sketch into a single zip file and upload to canvas.}



\end{document}
