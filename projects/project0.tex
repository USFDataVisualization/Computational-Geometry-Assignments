\documentclass[a4paper,12pt]{article}

\usepackage{../usfdvl}

\title{Project 0: Introduction to Processing}
\SetDocumentFooter{}{}


\begin{document}

\maketitle

\section{Objectives}

\myparagraph{This assignment will help you become familiar with Processing and build some basic data structures for storing geometric data.}


\projectGroundRules



\vspace{5pt}
\section{Assignment Instructions}

\begin{itemize}

\item Download and familiarize yourself with Processing (\url{http://processing.org/download/}). Use the available tutorials (\url{http://processing.org/tutorials/}) and examples (\url{http://processing.org/examples/}) to help you understand how Processing works.


\item Use the provided skeleton code to complete a sketch with the following requirements.

\begin{itemize}

\item When the mouse is clicked, a new point should be added to the point list. This point should be drawn using an ellipse and include a label.

\item Every point should be connected to the previously added point with an edge in the edge list. The edges list should be drawn every frame.

\item Every 3 points should form a triangle that is added to the triangle list. The triangles should be drawn and colored based upon whether they are stored in a clockwise or counterclockwise order.

\end{itemize}

\end{itemize}


\begin{minipage}[t]{0.625\textwidth}
\vspace{-145pt}
\projectSubmission
\end{minipage}
\hfill
%
\begin{minipage}[t]{0.3\textwidth}
\begin{center}
\includegraphics[width=0.975\linewidth]{../images/project0.png}
Example output.
\end{center}
\end{minipage}



\end{document}
