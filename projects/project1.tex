\documentclass[a4paper,12pt]{article}

\usepackage{../usfdvl}


\title{Project 1: Line Segment Intersections}
\SetDocumentFooter{}{}


\begin{document}

\maketitle

\projectGroundRules

\projectSubmission

\assignmentInstructions

\myparagraph{In this assignment you will implement some of the basic algorithms of line segment intersections. The will also have the opportunity to work on optimizing the performance of your code.}

\begin{itemize}

\item Download the provided skeleton code and complete the empty functions in \texttt{AABB.pde}, \texttt{Point.pde}, \texttt{Edge.pde}, and \texttt{LineSegmentSet.pde}. Of the highest importance:

\begin{itemize}
   \item \texttt{boolean AABB::intersectionTest( AABB other )} --- This function returns the result of an intersection test (true or false) for 2 axis aligned bounding boxes.
   
   \item \texttt{Point Edge::intersectionPoint( Edge other )} --- This function returns the exact intersection point between 2 line segments. You should use the parametric intersection method, with the emphasis on CORRRECTNESS of the result it produces.
   
   \item \texttt{Point Edge::optimizedIntersectionPoint( Edge other )} --- This function returns the exact intersection point between 2 line segments. You can use whatever intersection method you like, with the emphasis on SPEED.
   
\item \texttt{public static void OptimizedLineSegmentSetIntersectionAABB( 
ArrayList<Edge>} \linebreak
\texttt{input\_edges, ArrayList<Point> output\_intersections )} --- Given a set of input segments, this function finds all intersections between those segments. This method should use the axis aligned bounding box method discussed in class. However, you do NOT need to use an interval tree.

%\item \texttt{public static void OptimizedLineSegmentSetIntersection( ArrayList<Edge> } \linebreak
%\texttt{input\_edges, ArrayList<Point> output\_intersections )} --- Given a set of input segments, this function finds all intersections between those segments. 
%This function should be optimized in some way (you don't need to implement the full line sweep algorithm, but maybe take some inspiration). 
%Pairwise intersections should tested using the \texttt{optimizedIntersectionPoint()} function. Results should be correct, but the emphasis is on SPEED.
   
\end{itemize}

\item To help test your code, the skeleton code will provide random line segments and show the intersections points your code finds. You can change the number of line segments by using the '+'/'--' keys while running your sketch. We also provided 2 functionality for comparing naive and optimized versions of your algorithms:


\begin{itemize}

   \item \texttt{performanceTest()} (press 'p' when running your sketch) --- This function will run a series of random line segments sets through your algorithms and compare the performance results.

   \item \texttt{compareOutput()} (press 'c' when running your sketch) --- This function will run a series tests that check if the output of the algorithms are \textit{consistent} (not necessarily correct).

   
\end{itemize}
\end{itemize}

\end{document}
