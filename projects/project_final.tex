\documentclass[a4paper,12pt]{article}

\usepackage{../usfdvl}

\title{Final Project}
\SetDocumentFooter{}{}


\begin{document}

\maketitle

\section{Objectives}

\myparagraph{For this assignment is the you will \underline{propose}, \underline{implement}, \underline{experiment or demonstrate}, and \underline{report results} for one or more algorithms. The goal is that you provide some new information to the rest of the class.}


\vspace{5pt}
\section{Ground Rules}

\myparagraph{This assignment can be done in pairs. Pairs are expected to select a topic approximately twice as extensive as individuals. However, the code and reporting must be yours.}


\vspace{5pt}
\section{Assignment Instructions}

\myparagraph{The point of the project is to propose and execute some kind of experiment (i.e., comparison between algorithms) or demonstration of an algorithm.}

\myparagraph{3 deadlines will appear in Canvas. (1) Project proposal; (2) Mid-Project Report; (3) Final project submission. You will need to submit/upload the respective text, code, images, etc. for each.}

\begin{itemize}
    \item \textbf{Project Proposal} -- The project proposal will be 2 paragraphs describing your planned project. The first paragraph will describe the topic. Be explicit---for example, if you are implementing 3 algorithms, tell me which 3 and why you picked them. The second paragraph should describe what experimentation or demonstration you plan to perform. Again, be explicit---if you plan performance comparison, describe what types of comparisons you will do. If you plan to demonstrate something, describe exactly how (like what types of input data you will use). I only want 1 copy per team, but make sure you tell me who your partner is.
    
    \item \textbf{Mid-Project Report} -- By the time the mid-project report is due, you should be done or close to done with your code. For this report, you will need to provide a copy of your code with instructions for compiling/running and a 2 paragraph status report. The first paragraph will provide detail about the status of the development and experimentation/demonstration. The second paragraph will describe any changes in the scope of your project.
    
    \item \textbf{Final Project Submission} -- Your final submission will include your final code and a PowerPoint presentation given to the class. For the presentation, you will be allotted 5 minutes in class. You should produce a presentation of approximately 4-5 slides. In the slides, you will introduce the project, describe any experimentation, and discuss results and conclusions. 
        
\end{itemize}


\section{Potential Topics}

\myparagraph{The choice of topic is yours. Below is a suggested but not exclusive topic list. Important note: We can't have topics that overlap too much, and topics are first come, first serve. If this is discovered, students with overlapping topics will be asked to revise their planned projects.}

\begin{itemize}
    \item Implement three algorithms (or fewer, if non-trivial) for a specific topic (triangulation; convex hulls; point searches; etc.).
    \item Implement one (non-trivial) algorithm in 3 or more programming languages (Python, Java, and C for example). The programming languages should have significant enough distinctions (interpreted vs GC/JIT compiled vs native compiled) that will potentially result in performance differences.
    \item Implement incremental Voronoi construction algorithm.
    \item Calculate geodesic distance using a relative neighborhood graph and compare to Euclidean distance.
    \item Multiple ($>2$) convex polygon intersections, unions, differences, etc.
    \item Write a Processing sketch that animates any three (or fewer, if non-trivial) algorithms in your book. 
    \item Find the intersection of halfplanes.
    \item Form a robust distance library (point-point, point-line, point-polygon, line-polygon, etc.).
    \item Implement the art gallery problem solution discussed in class. Enable checking if user selected guard locations cover the polygon.
    \item Compute and Voronoi diagram and Delaunay triangulation of a point set.
    \item Implement and compare 2 clustering algorithms*.
	\item Implement and compare 2 spatial subdivision strategies*.
	\item Implement and compare 2 nearest neighbor strategies*.
    \item Other ideas are welcome (see \url{http://www.cse.usf.edu/~sarkar/cGeom/Computational_Geometry/Project.html} for more examples)
\end{itemize}



\end{document}
