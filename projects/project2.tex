\documentclass[a4paper,12pt]{article}

\usepackage{../usfdvl}


\title{Project 2: Polygon Diagonals}
\SetDocumentFooter{}{}


\begin{document}

\maketitle


\projectGroundRules

\projectSubmission

\assignmentInstructions

\myparagraph{In this assignment you will implement some of the basic polygon functionality.}

\begin{itemize}

\item Download the provided skeleton code and complete the unfinished functions in \texttt{Polygon.pde}.

\begin{itemize}

   \item \texttt{boolean Polygon::isSimple( )} --- This function checks to see that the boundary of the polygon is simple.
   
   \item \texttt{boolean Polygon::pointInPolygon( Point p )} --- Returns true if the point p is inside of the polygon.
   
   \item \texttt{ArrayList<Edge> Polygon::getDiagonals( )} --- Returns all of the valid diagonals for the polygon.
   
   \item \texttt{boolean Polygon::ccw( )} --- Returns true if the polygon is oriented counterclockwise.
   
   \item \texttt{boolean Polygon::cw( )} --- Returns true if the polygon is oriented clockwise.
         
   \item \texttt{float Polygon::area( )} --- Returns the area, in pixel, of the polygon.
   
\end{itemize}

\item To test your code the Processing skeleton provided gives visual feedback for creating polygons and testing your capabilities.

\end{itemize}




\end{document}
