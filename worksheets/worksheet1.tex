\documentclass[a4paper,12pt]{article}

\usepackage{../usfdvl}


\title{Worksheet 1}
\SetDocumentFooter{}{}


\begin{document}

\maketitle

\worksheetGroundRules

\worksheetSubmission

\vspace{3pt}
\section{Assignment} 



\begin{enumerate}

\item For each of the following triangles, where $\{p_1,p_2,p_3\}$:

\vspace{-10pt}
$$T_1=\{\{2,3\},\{5,6\},\{3,5\}\}$$
$$T_2=\{\{3,2\},\{1,6\},\{4,4\}\}$$
$$T_3=\{\{1,3\},\{5,9\},\{3,6\}\}$$

%\begin{itemize}
%\item $T_1=\{\{2,3\},\{5,6\},\{3,5\}\}$
%\item $T_2=\{\{3,2\},\{1,6\},\{4,4\}\}$
%\item $T_3=\{\{1,3\},\{5,9\},\{3,6\}\}$
%\item $\{\{1,4\},\{5,6\},\{9,8\}\}$
%\end{itemize}

\begin{enumerate}
\item Draw the triangles

\item Calculate the vectors: $\overrightarrow{A} = \overrightarrow{p_1p_2}; \overrightarrow{B} = \overrightarrow{p_1p_3}$

\item Calculate the angle between the following 2 vectors, $\angle\overrightarrow{A}\overrightarrow{B}$ using the:

\begin{itemize}
\item difference between 2 angles approach
\item dot product approach
\item cross product approach
\end{itemize}

\item Determine the orientation (clockwise or counterclockwise) of the following triangles.

\end{enumerate}



\item Perform the following linear interpolations:

\vspace{-10pt}
%\begin{itemize}
%\item $A=2; B=5; \alpha=0.4$
%\item $A=\{1,6\}; B=\{4,4\}; \alpha=0.7$
%\item $A=\{3,6\}; B=\{5,9\}; \alpha=0.2$
%\end{itemize}
$$A=2; B=5; \alpha=0.4$$
$$A=\{1,6\}; B=\{4,4\}; \alpha=0.7$$
$$A=\{5,6\}; B=\{5,9\}; \alpha=0.2$$



\item Answer the following questions about complexity:

\begin{itemize}
	\item Does $2^{n+1}=O(2^n)$? If not, what does it equal?
	\item Does $2^{2n}=O(2^n)$? If not, what does it equal?
	\item Generally speaking, what is the complexity of an algorithm that runs a single loop through $n$ data?
	\item Generally speaking, what is the complexity of an algorithm with nested loops, each running through $n$ data?
	\item Generally speaking, what is the complexity of tree operations (insert, remove, search) on a binary search tree? On a self-balanced binary tree?
\end{itemize}

%\item Prove by mathematical induction that the following formula, $3^2+3^3+...3^n=9\left( \frac{3^{n-1}-1}{2} \right)$, holds for $\forall \geq 2$.


\end{enumerate}




\end{document}
